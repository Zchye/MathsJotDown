\documentclass[12pt]{article}
\usepackage{hyperref} % This package is for creating hyper link-style corss references.
\usepackage{amsmath}
\usepackage{amsthm}
\usepackage{amsfonts}
\title{Optimal Wideband Precoder For MIMO}

\newtheorem{lemma}{Lemma}
\newtheorem{theorem}{Theorem}

\theoremstyle{definition}
\newtheorem{definition}{Definition}

\begin{document}
\maketitle
For a MIMO wireless communication, we have sampled transmit signal $x[n]\in\mathbb{C}^M$, receive signal $y[n]\in\mathbb{C}^N$, and a channel $h[n]\in\mbox{Hom}\left(\mathbb{C}^M, \mathbb{C}^M\right)$ all with lengths $P$
$$y[n]=\sum_{n-0}^{P-1}h[n]x[n]$$
And in the frequency domain we have
$$Y[k]=H[k]X[k]$$
where $k=0,\dots,P-1$.

We want to find an approximation of $h[n]$ denoted by $\hat{h}$ such that $y$ and $\hat{h}x$ are as close as possible, or, $Y$ and $\hat{h}X$ as close as possible. Later we will show that these two requirements are equivalent.

Note that $y$ and $Y$ are elements of a $\mbox{Hom}\left(\mathbb{C}^N \right)$-module, $h$ and $H$ are elements of another $\mbox{Hom}\left(\mathbb{C}^N \right)$-module, and $x$ as well as $X$ are elements of a $\mathbb{C}^{M\times M}$-module. To measure how close to elements of a module are, we can define an "inner product" analogy to the usual inner product defined on Hilbert spaces.
\begin{definition}[Inner product on $\mbox{Hom}\left(\mathbb{C}^N \right)$-modules]\label{def inner product}
	An auto-correlation on a $P$-dimensional free $\mbox{Hom}\left(\mathbb{C}^N \right)$-module $\mathcal{M}$ is a operation $\langle\cdot,\cdot\rangle$: $\mathcal{M}\times\mathcal{M}\rightarrow\mbox{Hom}\left(\mathbb{C}^N \right)$
	$$\langle x,y\rangle=\sum_{n=0}^{P-1}x[n]y[n]^*$$
	We also refer $C_{xy}$ to $\langle x,y\rangle$ for convenience under some context.
\end{definition}
\begin{definition}[Orthogonality on modules]\label{def orthogonality}
	$x,y\in\mathcal{M}$ are orthogonal if
	$$\langle x,y\rangle=0$$
\end{definition}
It is easy to verify that $\langle x,x\rangle$ is positive semi-definite. Therefore we can define the norm induced by the inner product
\begin{definition}[Euclidean norm on modules]\label{def norm}
	The Euclidean norm of $x\in\mathcal{M}$ is 
	$$||x||=\langle x,x\rangle^{1/2}$$
\end{definition}
It is obvious that the norm of any element is positive semi-definite. Hence, there is a partial order $\preceq$ induced by the positive semi-definite cone. And we are able to measure the closeness of two elements in a module by this partial order.

The theorem follows tells us that we can study the closeness in either time domain or frequency domain.
\begin{theorem}[Extended Parseval's Theorem]\label{thm parseval}
	$$||X||=\sqrt{P}||x||$$
	w.r.t. the DFT defined by
	$$X[k]=\sum_{n=0}^{P-1}x[n]\omega_P^{nk}$$
	where
	$$\omega_P^{nk}=e^{-j2\pi nk/P}$$
\end{theorem}
\begin{proof}
	\begin{eqnarray*}
	||X||^2&=&\sum_{k=0}^{P-1}X[k]X[k]^*\\
	&=&\sum_{k=0}^{P-1}\left(\sum_{m=0}^{P-1}x[m]\omega_P^{mk} \right)\left(\sum_{n=0}^{P-1}x[n]\omega_P^{nk} \right)^*\\
	&=&\sum_{k=0}^{P-1}\sum_{m=0}^{P-1}\sum_{n=0}^{P-1}x[m]x[n]^*\omega_P^{(m-n)k}
	\end{eqnarray*}
Note that the terms being summed are non-zero only if $m=n$, thus
$$||X||^2=\sum_{k=0}^{P-1}Px[k]x[k]^*=P||x||^2$$
completes the proof.
\end{proof}
To find the optimal $\hat{h}$ in time domain is equivalent to find it in the frequency domain. This amounts to say that we are trying to approximate the frequency-selective $H[k]$ with a constant $\hat{h}$ over all frequencies.

We need to extend the orthogonality principle to modules. Before that we need another kind of product of elements from different modules.
\begin{definition}[Cross-correlation]\label{def crosscorrelation}
	The cross-correlation is a binary operator defined on the product of two modules $\langle\cdot,\cdot\rangle$: $\mathcal{N}\times \mathcal{M}\rightarrow\mbox{Hom}\left(\mathbb{C}^M,\mathbb{C}^N \right)$
	$$\langle x,y\rangle=\sum_{n=0}^{P-1}x[n]y[n]^*$$
	And we sometimes refer $C_{xy}$ to $\langle x,y\rangle$ for convenience under certain contexts.
\end{definition}
When $N=M$, the cross-correlation coincides with auto-correlation.
\begin{lemma}\label{lemma |z+mux|>=|z|}
	Let $z\in\mathcal{N}$, $x\in\mathcal{M}$ s.t. $C_{zx}=0$. Then $\forall \mu\in\mbox{Hom}\left(\mathbb{C}^M,\mathbb{C}^N \right)$ we have
	$$||z+\mu x||\succeq||z||$$
\end{lemma}
\begin{proof}
	It is easy to verify that
	$$||z+\mu x||^2=||z||^2+||\mu x||^2$$
	and we are done.
\end{proof}
It follows from lemma \ref{lemma |z+mux|>=|z|} that every $y\in\mathcal{N}$ can be decomposed into two parts
$$y=z+\mu x$$
where $C_{zx}=0$ and $\mu\in\mbox{Hom}\left(\mathbb{C}^M, \mathbb{C}^N\right)$
\end{document}