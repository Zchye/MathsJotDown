\documentclass[12pt]{article}
\usepackage{hyperref} % This package is for creating hyper link-style corss references.
\usepackage{amsmath}
\usepackage{amsthm}
\usepackage{mathrsfs}
\title{Rank Of Sum Of Positive Semidefinite Matrices}

\newtheorem{lemma}{Lemma}
\begin{document}
	\maketitle
	We want to proof 
	\begin{equation}
		\mbox{rank}(A+B)\geq\max(\mbox{rank}(A),\mbox{rank}(B)) \label{r(A+B)>=rA+rB}
	\end{equation}
	provided that $A,B\in \mbox{S}_+^N$, i.e., positive semidefinite matrices of order $N$.

\begin{lemma}\label{lemma rank(S+I)=N}
	$\forall S\in\mbox{S}_+^N$
	\begin{equation*}
		\mbox{rank}(S+I_N)=N
	\end{equation*}
\end{lemma}
\begin{proof}
	Apply spectral decomposition on $S$, $\exists U\in \mbox{U}(N)$ (group of unitary matrices) and diagonal matrices $D$ with nonnegative entries, such that
	$$
	S+I=U(D+I)U^*
	$$
	where the diagonal entries of $D+I$ are strictly positive. It follows from that unitary matrices do not change rank.
\end{proof}
\begin{lemma}\label{lemma A=PP*}
	$\forall A\in\mbox{S}_+^N$, $\exists P\in\mbox{U}(r)$ s.t. 
	$$
	\tilde{A}=PP^*
	$$
	where $r=\mbox{rank}(A)$ and $\tilde{A}=A|_{V/\ker A}$.
\end{lemma}
\begin{proof}
	First apply spetral decopomsition on $\tilde{A}$
	$$ \tilde{A} = UDU^* $$
	Then let $L=\sqrt{D}$ and $P=UL$
	completing the proof.
\end{proof}

Now we are able to prove the inequality of interest.
\begin{proof}
	(Rank inequality of sum of positive semidefinite matrices)
	
	Suppose $A$ is non-trivial, otherwise the inequality trivially holds. Restrict $A$ and $B$ on $V/\ker A$ to get $\tilde{A}$ and $\tilde{B}$. Then apply lemma \ref{lemma A=PP*}
	\begin{equation}
		\tilde{A}+\tilde{B}=P(I+P^*\tilde{B}P)P^*
		\label{eqn SVD(A+B)}
	\end{equation}
	It follows from lemma \ref{lemma rank(S+I)=N} that
	$$\mbox{rank} (\tilde{A}+\tilde{B})=\mbox{rank}(I)=\mbox{rank}(\tilde{A})$$
	Since $A+B$ might not be trivial on $\ker A$, we have
	$$\mbox{rank}(A+B)\geq \mbox{rank}(\tilde{A}+\tilde{B})$$
	Also note that
	$$\mbox{rank}(A)= \mbox{rank}(\tilde{A})$$
	and we are done.
\end{proof}
Equation \ref{eqn SVD(A+B)} also provides us a means to study the relationship between the spectral decomposition of (A+B) and of A and B respectively.

We have a more elegant proof though, by introducing a lemma first
\begin{lemma}\label{lemma kerA}
	$\forall v\in V$, $v^*Av=0\Leftrightarrow v\in\ker A$
\end{lemma}
\begin{proof}
	$\Leftarrow$ is obvious. 
	
	$\Rightarrow$
	follows from lemma \ref{lemma A=PP*}, if $\tilde{A}$ is non-trivial, then it is positive definite. This completes the proof.
\end{proof}
Now the proof follows
\begin{proof}
	$\forall A\in \mbox{S}_+^N$, define $\mathscr{N}(A)$ to be
	$$\mathscr{N}(A)=\{v\in V|v^*Av=0 \}$$
	It is easy to verify that
	$$\mathscr{N}(A+B)=\mathscr{N}(A)\cap \mathscr{N}(B)$$
	Apply lemma \ref{lemma kerA}
	$$\ker(A+B)=\ker A\cap\ker B\subseteq\ker A$$
	which implies
	$$\mbox{rank}(A+B)\geq\mbox{rank}(A)$$
	which completes the proof.
\end{proof}
\end{document}